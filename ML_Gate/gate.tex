\documentclass[a4paper,12pt]{article}
\usepackage{circuitikz}
\usepackage{tikz}
\usepackage{amsmath, amssymb, graphicx}

\begin{document}

\begin{center}
    {\Large \textbf{GATE QUESTIONS}} \\
    {\large \textbf{EC PAPER - 2010}} \\
    \vspace{0.5cm}
    \textbf{MAHALAKSHMI. I. SUNKAD} \\
    \textbf{COMETFWC012} \\
\end{center}

\vspace{0.5cm}

\noindent\textbf{Q.12)} For the output F to be 1 in the logic circuit shown, \\
the input combination should be:

\vspace{0.5cm}

\textbf{Logic Circuit Diagram}

%\begin{figure}[h]
%    \centering
   % \includegraphics[width=0.8\textwidth]{circuit_diagram.png} % Change filename accordingly
  %  \caption{Logic Circuit Diagram}
 %   \label{fig:circuit}
%\end{figure}

\begin{center}
    \begin{circuitikz}
        % Inputs
        \node (A) at (-1.5,2.25) {\textbf{A}};
        \node (B) at (-1,1.5) {\textbf{B}};
        \node (C) at (2,0.5) {\textbf{C}};
        
        % First XOR Gate
        \node [xor port, scale=1.2] (XOR1) at (2,2) {};
        \draw (A.east) -- (XOR1.in 1);
        \draw (B.east) -- (XOR1.in 2);
        \node at (1, 3) {\small $X = A \oplus B$};
        
        % Second XNOR Gate
        \node [xnor port, scale=1.2] (XNOR1) at (2,0) {};
        \draw (A.east) |- (XNOR1.in 1);
        \draw (B.east) |- (XNOR1.in 2);
        \node at (2, -1) {\small $Y = A \odot B$};

        % Third 3-input XNOR Gate
        \node [xnor port, scale=1.2] (XNOR2) at (7,1) {};
        \draw (XOR1.out) -- ++(1,0) |- (XNOR2.in 1);
        \draw (XNOR1.out) -- ++(1,0) |- (XNOR2);
        \draw (C.east) -- (XNOR2.in 2);     
        \node at (6, 2) {\small $F = \overline{X \oplus Y \oplus C}$};

        % Output
        \node (F) at (9,1) {\textbf{F}};
        \draw (XNOR2.out) -- (F.west);
    \end{circuitikz}
\end{center}
\vspace{0.5cm}
\noindent\textbf{(A)} A = 1, B = 1, C = 0 \\ 
\textbf{(B)}A = 1, B = 0, C = 0 \\ 
\textbf{(C)} A = 0, B = 1, C = 0 \\ 
\textbf{(D)} \bf{A = 0, B = 0, C = 1}
\section*{Solution}

\textbf{Step 1: Understanding the Logic Circuit}

The circuit consists of:

\begin{enumerate}
    \item \textbf{First Gate (XOR)}  
    \begin{itemize}
        \item Inputs: $A, B$
        \item Output: $X$
    \end{itemize}
    
    \item \textbf{Second Gate (XNOR)}  
    \begin{itemize}
        \item Inputs: $A, B$
        \item Output: $Y$
    \end{itemize}
    
    \item \textbf{Third Gate (3-input XNOR)}  
    \begin{itemize}
        \item Inputs: $X, Y, C$
        \item Output: $F$
        \end{itemize}
\end{enumerate}

\hrulefill

\textbf{Step 2: Deriving Boolean Expressions}

\begin{enumerate}
    \item \textbf{First XOR Gate:}
\[
    X = A \oplus B = A\overline{B} + \overline{A}B
\]
    
    \item \textbf{Second XNOR Gate:}
\[
    Y = A \odot B = AB + \overline{A} \overline{B}
\]

    \item \textbf{Third XNOR Gate (3-input XNOR):}
\[
    F = X \odot Y \odot C
\]
\[
    F = \overline{X \oplus Y \oplus C}
\]
\[
    F = \overline{(A \oplus B) \oplus (A \odot B) \oplus C}
\]
\end{enumerate}

\hrulefill

\textbf{Step 3: Constructing the Truth Table}

\begin{center}
    \begin{tabular}{|c|c|cc|cc|}
        \hline
        A & B & C & X (XOR) & Y (XNOR) & F (3-input XNOR) \\ 
        \hline
        0 & 0 & 0 & 0 & 1 & 1 \\  
        0 & 0 & 1 & 0 & 1 & 0 \\  
        0 & 1 & 0 & 1 & 0 & 1 \\  
        0 & 1 & 1 & 1 & 0 & 0 \\  
        1 & 0 & 0 & 1 & 0 & 1 \\  
        1 & 0 & 1 & 1 & 0 & 0 \\  
        1 & 1 & 0 & 0 & 1 & 1 \\  
        1 & 1 & 1 & 0 & 1 & 0 \\  
        \hline
    \end{tabular}
\end{center}
\hrulefill

\section*{Step 4: Finding the Correct Answer}

For $F = 1$, the valid input combination is:

\[
(0,0,1)
\]

Comparing with given options:

\begin{itemize}
    \item (A) $A = 1, B = 1, C = 0$
    \item (B) $A = 1, B = 0, C = 0$
    \item (C) $A = 0, B = 1, C = 0$ 
    \item (D) $A = 0, B = 0, C = 1$  (Correct)
\end{itemize}

\textbf{Final Answer:}  

\[
\textbf{Option (D): A = 0, B = 0, C = 1}
\]

\end{document}
